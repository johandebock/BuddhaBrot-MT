\documentclass[10pt,a4paper]{article}

\usepackage[margin=1cm]{geometry}
\usepackage{tabularx}
\newcolumntype{R}{>{\raggedleft\arraybackslash}X}
\newcolumntype{L}{>{\raggedright\arraybackslash}X}
\usepackage{booktabs}



\begin{document}
\begin{center}{\Huge BuddhaBrot-MT manual}\end{center}

\begin{table}[h!]
    \caption{Changing layer mode, changing color table type, changing BuddhaBrot type (0=BuddhaBrot, 1=Anti-Buddhabrot, 2=Anti-Buddhabrot with some lobes cut)}
    \setlength{\tabcolsep}{0.0pt}
    \begin{tabularx}{\linewidth}{lRRR}
        \toprule
                   & F1                & F2                     & `                     \\
        \midrule                                                
        -          & cycle layer mode  & cycle color table type & cycle BuddhaBrot type \\
        Shift      &                   &                        &                       \\
        Ctrl       &                   &                        &                       \\
        Shift+Ctrl &                   &                        &                       \\
        \bottomrule
    \end{tabularx}
\end{table}

\begin{table}[h!]
    \caption{Saving, loading, calculation thread handling, changing animation frame rate}
    \setlength{\tabcolsep}{0.0pt}
    \begin{tabularx}{\linewidth}{lRRRR}
        \toprule
                   & F9                & F10                     & F11               & F12    \\
        \midrule
                   & save status      & load status             & pause calculations & 1 fps  \\
        Shift      & save parameters  & load parameters         & threads $+$= 3     & 10 fps \\
        Ctrl       &                  & load status (threads=3) & threads $-$= 3     & 30 fps \\
        Shift+Ctrl &                  &                         &                    &        \\
        \bottomrule
    \end{tabularx}
\end{table}

\begin{table}[h!]
    \caption{Changing BuddhaBrot parameter: bailout (bail)}
    \setlength{\tabcolsep}{0.0pt}
    \begin{tabularx}{\linewidth}{lRRRR}
        \toprule
                   & 1                      & q                    & a                    & z                    \\
        \midrule                                                                          
        -          & layer 123 bail $+$= 1  & layer 1 bail $+$= 1  & layer 2 bail $+$= 1  & layer 3 bail $+$= 1  \\
        Shift      & layer 123 bail $*$= 10 & layer 1 bail $*$= 10 & layer 2 bail $*$= 10 & layer 3 bail $*$= 10 \\
        Ctrl       & layer 123 bail $-$= 1  & layer 1 bail $-$= 1  & layer 2 bail $-$= 1  & layer 3 bail $-$= 1  \\
        Shift+Ctrl & layer 123 bail $/$= 10 & layer 1 bail $/$= 10 & layer 2 bail $/$= 10 & layer 3 bail $/$= 10 \\
        \bottomrule
    \end{tabularx}
\end{table}

\begin{table}[h!]
    \caption{Changing BuddhaBrot parameter: path plot start (pps)}
    \setlength{\tabcolsep}{0.0pt}
    \begin{tabularx}{\linewidth}{lRRRR}
        \toprule
                   & 2                      & w                    & s                    & x                    \\
        \midrule                                                                          
        -          & layer 123 pps $+$= 1  & layer 1 pps $+$= 1  & layer 2 pps $+$= 1  & layer 3 pps $+$= 1  \\
        Shift      & layer 123 pps $*$= 10 & layer 1 pps $*$= 10 & layer 2 pps $*$= 10 & layer 3 pps $*$= 10 \\
        Ctrl       & layer 123 pps $-$= 1  & layer 1 pps $-$= 1  & layer 2 pps $-$= 1  & layer 3 pps $-$= 1  \\
        Shift+Ctrl & layer 123 pps $/$= 10 & layer 1 pps $/$= 10 & layer 2 pps $/$= 10 & layer 3 pps $/$= 10 \\
        \bottomrule
    \end{tabularx}
\end{table}

\begin{table}[h!]
    \caption{Changing BuddhaBrot parameter: path plot end (ppe)}
    \setlength{\tabcolsep}{0.0pt}
    \begin{tabularx}{\linewidth}{lRRRR}
        \toprule
                   & 3                      & e                    & d                    & c                    \\
        \midrule                                                                          
        -          & layer 123 ppe $+$= 1  & layer 1 ppe $+$= 1  & layer 2 ppe $+$= 1  & layer 3 ppe $+$= 1  \\
        Shift      & layer 123 ppe $*$= 10 & layer 1 ppe $*$= 10 & layer 2 ppe $*$= 10 & layer 3 ppe $*$= 10 \\
        Ctrl       & layer 123 ppe $-$= 1  & layer 1 ppe $-$= 1  & layer 2 ppe $-$= 1  & layer 3 ppe $-$= 1  \\
        Shift+Ctrl & layer 123 ppe $/$= 10 & layer 1 ppe $/$= 10 & layer 2 ppe $/$= 10 & layer 3 ppe $/$= 10 \\
        \bottomrule
    \end{tabularx}
\end{table}

\begin{table}[h!]
    \caption{Changing BuddhaBrot parameter: path minimum n\_inf (minn)}
    \setlength{\tabcolsep}{0.0pt}
    \begin{tabularx}{\linewidth}{lRRRR}
        \toprule
                   & 4                      & r                    & f                    & v                    \\
        \midrule                                                                          
        -          & layer 123 minn $+$= 1  & layer 1 minn $+$= 1  & layer 2 minn $+$= 1  & layer 3 minn $+$= 1  \\
        Shift      & layer 123 minn $*$= 10 & layer 1 minn $*$= 10 & layer 2 minn $*$= 10 & layer 3 minn $*$= 10 \\
        Ctrl       & layer 123 minn $-$= 1  & layer 1 minn $-$= 1  & layer 2 minn $-$= 1  & layer 3 minn $-$= 1  \\
        Shift+Ctrl & layer 123 minn $/$= 10 & layer 1 minn $/$= 10 & layer 2 minn $/$= 10 & layer 3 minn $/$= 10 \\
        \bottomrule
    \end{tabularx}
\end{table}

\begin{table}[h!]
    \caption{Changing coloring method (cm) (0=rank-order mapping, 1=histogram mapping, 2=log+rank-order mapping, 3=log+histogram mapping), changing logarithmic offset for coloring methods 23 (log)}
    \setlength{\tabcolsep}{0.0pt}
    \begin{tabularx}{\linewidth}{lRRRR}
        \toprule
                   & 5                    & t                  & g                  & b                  \\
        \midrule                                                                          
        -          & layer 123 cycle cm   & layer 1 cycle cm   & layer 2 cycle cm   & layer 3 cycle cm   \\
        Shift      & layer 123 log $+$= 1 & layer 1 log $+$= 1 & layer 2 log $+$= 1 & layer 3 log $+$= 1 \\
        Ctrl       & layer 123 log $-$= 1 & layer 1 log $-$= 1 & layer 2 log $-$= 1 & layer 3 log $-$= 1 \\
        Shift+Ctrl &                      &                    &                    &                    \\
        \bottomrule
    \end{tabularx}
\end{table}

\begin{table}[h!]
    \caption{Changing color table offset (ct\_o)}
    \setlength{\tabcolsep}{0.0pt}
    \begin{tabularx}{\linewidth}{lRRRR}
        \toprule
                   & 6                       & y                     & h                     & n                     \\
        \midrule                                                                          
        -          & layer 123 ct\_o $+$= 1  & layer 1 ct\_o $+$= 1  & layer 2 ct\_o $+$= 1  & layer 3 ct\_o $+$= 1  \\
        Shift      & layer 123 ct\_o $+$= 10 & layer 1 ct\_o $+$= 10 & layer 2 ct\_o $+$= 10 & layer 3 ct\_o $+$= 10 \\
        Ctrl       & layer 123 ct\_o = 0     & layer 1 ct\_o = 0     & layer 2 ct\_o = 0     & layer 3 ct\_o = 0     \\
        Shift+Ctrl &                         &                       &                       &                       \\
        \bottomrule
    \end{tabularx}
\end{table}

\begin{table}[h!]
    \caption{Changing color table cycle speed (ct\_v)}
    \setlength{\tabcolsep}{0.0pt}
    \begin{tabularx}{\linewidth}{lRRRR}
        \toprule
                   & 7                      & u                    & j                    & m                    \\
        \midrule                                                                          
        -          & layer 123 ct\_v $+$= 1 & layer 1 ct\_v $+$= 1 & layer 2 ct\_v $+$= 1 & layer 3 ct\_v $+$= 1 \\
        Shift      & layer 123 ct\_v $-$= 1 & layer 1 ct\_v $-$= 1 & layer 2 ct\_v $-$= 1 & layer 3 ct\_v $-$= 1 \\
        Ctrl       & layer 123 ct\_v = 0    & layer 1 ct\_v = 0    & layer 2 ct\_v = 0    & layer 3 ct\_v = 0    \\
        Shift+Ctrl &                        &                      &                      &                      \\
        \bottomrule
    \end{tabularx}
\end{table}

\end{document}
